\documentclass[12pt, letterpaper]{article}
\usepackage{SyllabusStyle}

\begin{document}
\begin{center}
	{\Large \textsc{Physical Chemistry Lab II}}

	CHEM 3625
\end{center}

\begin{center}
	{\large Spring 2023}
\end{center}
\begin{center}
	\rule{0.99\textwidth}{0.4pt}
	\begin{tabular}{llcll}
		\textbf{Instructor:} & Matthew Rowley           &  & \textbf{Office Hours:} & Daily 10:00 am -- 11:00 am \\
		\textbf{Telephone:}  & (435) 586-7875           &  &                        &                            \\
		\textbf{Email:}      & matthewrowley$1$@suu.edu &  & \textbf{Office:}       & SC-220                     \\
		\multicolumn{5}{c}{Please include the course number in the subject line of all correspondence.}
	\end{tabular}
	\rule{0.99\textwidth}{0.4pt}
\end{center}


\section*{Course Description}
This course is the laboratory to accompany CHEM 3620 -- Physical Chemistry II. We will observe and explore chemical systems which clearly demonstrate principles of quantum mechanics.

\paragraph{Prerequisites:}
None

\paragraph{Concurrent requisite:}
CHEM 3620 -- Physical Chemistry II

\paragraph{Course Materials:} ~

\noindent No lab manual will be required. The instructions for each experiment will be posted on Canvas.

\noindent You will be required to have and wear your own pair of OSHA-approved safety goggles whenever you are in the lab. Students without eye protection will be required to leave the lab and will receive a zero for the labwork that day.

\paragraph{Student Learning Outcomes:}
\begin{description}
	\item[Knowledge of the physical and natural world] -- Students will recall, interpret, compare, explain, and apply chemistry terminology and theory.
	\item[Quantitative Literacy] -- Students will use chemical equations, graphs and tables to interpret and communicate chemical information.
	\item[Inquiry and Analysis] -- Students will solve complex chemical problems.
	\item[Communication] -- Students will report laboratory results clearly and concisely.
	\item[Problem Solving] -- Students will design and implement experimental procedures.
	\item[Teamwork] -- Students will productively interact with each other to successfully conduct chemistry experiments.
\end{description}

\section*{Laboratory Work}
Before lab, you are expected to have read the handout of your experiment as well as review your lecture notes from class. Come prepared to enter your data into the lab computers and have a USB drive with you. You may perform each laboratory with a lab partner and you may acquire your data together during your scheduled lab time. However, you must NOT work with your lab partner beyond this. All analysis of data and calculations as well as all laboratory reports must be done on an individual basis. Failure to do so will result in a zero for the lab in question.

\noindent Please follow all safety procedures, especially by wearing safety glasses or goggles. When leaving the lab, please make sure it is in the same condition as it was when you arrived. Be respectful of others.

\paragraph{Laboratory Risk:}
Chemical exposure is a constant risk in a chemistry lab. To minimize the risk to yourself and those around you, the following rules must be followed:
\begin{itemize}
	\item Never taste or smell a chemical or pipette by mouth.
	\item Wash your hands before leaving the lab and frequently during the lab to avoid accidental contamination of yourself and others.
	\item Dispose of chemicals only as directed. Nothing goes down the sink unless expressly directed.
	\item Keep your work area clean; wipe up any spills (liquid or solid) immediately.
	\item Replace caps on reagent bottles, and never return chemicals to the original container.
	\item No shorts, tank tops, or sandals allowed in lab, and long hair should be restrained.
	\item Wear safety glasses at all times when in the lab.
\end{itemize}
Students enrolling in this course should realize that they are voluntarily exposing themselves to a variety of chemicals, some of which could be irritating or hazardous with excessive exposure.  For those persons with a sensitive medical condition such as allergies, precautions such as wearing additional protective garments, delaying enrolling, or even not enrolling in a class may be necessary.  In particular, women who are their first trimester of pregnancy should avoid exposure to many chemicals unless approved by their physician.

\section*{Tentative Schedule}
This class will meet on Thursdays from 3:30pm -- 5:20pm in room 224 of the Science Center (SC)

\paragraph{Week 1: Jan. 9 -- Jan. 13}~

No Lab

\paragraph{Week 2: Jan. 16 -- Jan. 20}~

Group 1: Quantum Dots and Particle in a Box – A Spectroscopic Study

Group 2: Vibrational-Rotational Spectra of HCl/DCl
\paragraph{Week 3: Jan. 23 -- Jan. 27}~

Group 3: Quantum Dots and Particle in a Box – A Spectroscopic Study

Group 4: Vibrational-Rotational Spectra of HCl/DCl
% Exp 1: Quantum Dots and Particle in a Box – A Spectroscopic Study

\paragraph{Week 4: Jan. 30 -- Feb. 3}~

Group 1: Vibrational-Rotational Spectra of HCl/DCl

Group 2: Quantum Dots and Particle in a Box – A Spectroscopic Study
% No Lab -- Analyze Quantum Dot Data
\paragraph{Week 5: Feb. 6 -- Feb. 10}~

Group 3: Vibrational-Rotational Spectra of HCl/DCl

Group 4: Quantum Dots and Particle in a Box – A Spectroscopic Study
% Exp. 2: Absorption Spectrum of Conjugated Dyes

\paragraph{Week 6: Feb. 13 -- Feb. 17}~

Group 1: Absorption Spectrum of Conjugated Dyes

Group 2: Computational Applications of Quantum Chemistry
% No Lab -- Analyze Conjugated Dyes Data

\paragraph{Week 7: Feb. 20 -- Feb. 24}~

Group 3: Absorption Spectrum of Conjugated Dyes

Group 4: Computational Applications of Quantum Chemistry
% Exp. 3: Photophysics of Pyrene

\paragraph{Week 8: Feb. 27 -- Mar. 3}~

\textbf{No Classes this week (Spring Break!)}

\paragraph{Week 9: Mar. 6 -- Mar. 10}~

Group 1: Computational Applications of Quantum Chemistry

Group 2: Absorption Spectrum of Conjugated Dyes
% No Lab -- Analyze Pyrene Data

\paragraph{Week 10: Mar. 13 -- Mar. 17}~

Group 3: Computational Applications of Quantum Chemistry

Group 4: Absorption Spectrum of Conjugated Dyes
% Exp 4: Computational Applications of Quantum Chemistry

\paragraph{Week 11: Mar. 20 -- Mar. 24}~

Group 1: Photophysics of Pyrene

Group 2: Calculation of π-type Delocalized Orbitals Using Hückle Theory
% No Lab -- Analyze Computational Data

\paragraph{Week 12: Mar. 27 -- Mar. 31}~

Group 3: Photophysics of Pyrene

Group 4: Calculation of π-type Delocalized Orbitals Using Hückle Theory
% Exp. 5: Vibrational-Rotational Spectra of HCl/DCl

\paragraph{Week 13: Apr. 3 -- Apr. 7}~

Group 1: Calculation of π-type Delocalized Orbitals Using Hückle Theory

Group 2: Photophysics of Pyrene
% No Lab -- Analyze HCl/DCl Data

\paragraph{Week 14: Apr. 10 -- Apr. 14}~

Group 3: Calculation of π-type Delocalized Orbitals Using Hückle Theory

Group 4: Photophysics of Pyrene
% Exp. 6: Calculation of π-type Delocalized Orbitals Using Hückle Theory

\paragraph{Week 15: Apr. 17 -- Apr. 21}~

\textbf{Final Exam}

\paragraph{Finals Week}~

No Final -- You took it last week!

\section*{Course Requirements}
Grades for this class will be determined based on the following items:

\begin{description}
	\item[Pre-Lab Quizzes (10 points each)] -- Quizzes must be completed at the beginning of each lab. You may take the better score out of two attempts at these quizzes.
	\item[Lab Reports (40 points each)] -- Reports must be turned in at the beginning of the following lab.
	\item[Lab Final (100 points)] -- The final will be given on the last scheduled day of class (Dec. 1).
\end{description}

\noindent Final Grades will be assigned according to the following scale:

\begin{tabular}{cl|c|cl}
	Percentage & Grade &  & Percentage & Grade \\ \midrule
	100--93.0  & A     &  & 77.0--73.0 & C     \\
	93.0--90.0 & A-    &  & 73.0--70.0 & C-    \\
	90.0--87.0 & B+    &  & 70.0--67.0 & D+    \\
	87.0--83.0 & B     &  & 67.0--63.0 & D     \\
	83.0--80.0 & B-    &  & 63.0--60.0 & D-    \\
	80.0--77.0 & C+    &  & < 60.0     & F
\end{tabular}

\paragraph{Note that you must complete \emph{all} of the labs to pass this course!} Regardless of your other scores, an incomplete lab will result in an incomplete grade for entire course.

\paragraph{Report Grading:}
Lab reports will be graded on the quality of both their \emph{scientific content} and their \emph{presentation} in the following way:
\begin{description}
	\item[Presentation] -- Presentation includes writing quality, writing style, clarity, organization, formatting, grammar, etc.

	\item[Scientific Content] -- Your report should demonstrate a clear understanding of the basic principles at play in the lab. Data presentation should show an understanding of what the data mean and why they  are important (e.g. mislabeled axes show a lack of understanding). Analysis of your data and any conclusions drawn should have a sound basis in the scientific theories you have been taught.
\end{description}

Different chemistry journals have different formats and requirements. To keep things simple, your reports should have four sections:
\begin{description}
	\item[Introduction] -- Show background knowledge and general understanding of the experiment.
	\item[Method] -- Outline your specific procedure with enough detail that a competent chemist could reproduce your work.
	\item[Results] -- Results should include charts of any relevant data, as well as a description of any qualitative observations made in the course of the experiment.
	\item[Discussion] -- Interpret your results and draw conclusions. The lab manual will often prompt you with questions which must be answered in this section.
\end{description}

\paragraph{Late Work Policy:}
Laboratory reports must be turned in at or before the beginning of the following lab. There is a large window of time in which to analyze and write-up your results, so please plan to do the work early if you have any scheduling conflicts. Late work will not be accepted.

\paragraph{Make-up Work Policy:}
In general, there will be no opportunity to make up missed labs. This is particularly important because you must complete all of the labs to pass this course. If you must miss your assigned lab time, please arrange with your partner and me to do the lab some other time within the same week.


\section*{Miscellany}

\paragraph{Important syllabus statements related to ATTENDANCE and COVID-19:} ~

\noindent\emph{What should I expect in the classroom this semester?}

\noindent The following are general guidelines for the classroom environment
\begin{description}
	\item[Class Attendance is Required:] If you are registered for a Face-to-Face, Synchronous Remote, or Hybrid course, attendance is required. If you are ill or instructed to isolate or quarantine, you may request a faculty member record the class and share it with you or you may request other reasonable accommodations. Your instructor will work with you to develop a plan for completing coursework while you are isolated/quarantined. In order for you to receive academic accommodations and ensure that your request is communicated to faculty, you must submit this \href{https://my.suu.edu/covid/selfreport/}{self report form}.
	\item[\href{https://www.suu.edu/registrar/onlinehybrid.html}{Course ~delivery ~modalities} ~are ~posted ~online ~for ~each ~course, ~but ~may ~be ~modified ~in] \textbf{response to emerging COVID conditions:} SUU is employing every effort to maintain a learning environment that is engaging and safe. The course modality listed when you registered for courses should remain for the semester; however, due to COVID conditions, the delivery of modality for a specific course may change during the semester. Normally, these changes will be short term (possibly the length of a quarantine or isolation time period), or in some cases longer. When such a modification is needed, faculty members will work with their department chair and/or dean and the students to maintain an effective learning environment.
\end{description}

\paragraph{Scientific Calculator:}
There are many different ways to calculate figures during homework. It is tempting to rely on Online resources such as \href{http://www.wolframalpha.com}{http://www.wolframalpha.com}, or to simply use a calculator application on a smart phone. During exams, however, any devices capable of connecting to the Internet will \emph{not} be allowed. You will instead need a scientific calculator capable of performing exponentiation and logarithms for the exams. Using this calculator exclusively while doing homework will ensure that you are familiar with it for use during exams.

\paragraph{Academic Integrity:}
Scholastic dishonesty will not be tolerated and will be prosecuted to the fullest extent (see \href{https://www.suu.edu/policies/06/33.html}{SUU Policy 6.33}). You are expected to have read and understood the current SUU student conduct code (\href{https://www.suu.edu/policies/11/02.html}{SUU Policy 11.2}) regarding student responsibilities and rights, the intellectual property policy (\href{https://www.suu.edu/policies/05/52.html}{SUU Policy 5.52}), information about procedures, and what constitutes acceptable behavior.

\paragraph{Mental Health:}
Mental and physical health are equal components to a holistic view of wellness and human thriving. Mental health should not be ignored, dismissed, or demeaned. If you find yourself struggling with mental health please visit \href{https://www.suu.edu/mentalhealth}{https://www.suu.edu/mentalhealth} for resources. There is also a link prominently on the right side of every Canvas page.

\paragraph{ADA Policy:}
Students with medical, psychological, learning, or other disabilities desiring academic adjustments, accommodations, or auxiliary aids will need to contact the Southern Utah University Coordinator of Services for Students with Disabilities (SSD), in Room 206F of the Sharwan Smith Center or phone (435) 865-8022. SSD determines eligibility for and authorizes the provision of services.

\paragraph{Emergency Management Statement:}
In case of emergency, the university's Emergency Notification System (ENS) will be activated. Students are encouraged to maintain updated contact information using the link on the homepage of the \emph{mySUU} portal. In addition, students are encouraged to familiarize themselves with the Emergency Response Protocols posted in each classroom. Detailed information about the university's emergency management plan can be found at: \href{http://www.suu.edu/emergency}{http://www.suu.edu/emergency}

\paragraph{HEOA Compliance Statement:}
The sharing of copyrighted material through peer-to- peer (P2P) file sharing, except as provided under U.S. copyright law, is prohibited by law. Detailed information can be found at: \href{https://help.suu.edu/article/1097/p2p-and-copyright-infringement}{https://help.suu.edu/article/1097/p2p-and-copyright-infringement}

\paragraph{LINK Statement:}
SUU faculty and staff care about the success of our students. In addition to your professor, numerous services are available to assist you with the achievement of your educational goals. SUU's LINK system may be used by faculty to notify you and/or your advisors of their concern for your progress and provide references to campus services as appropriate.

\paragraph{SUUSA Statement:}
As a student at SUU, you have representation from the SUU Student Association (SUUSA) which advocates for student interests and helps work as a liaison between the students and the university administration. You can submit My SUU Voice feedback by going here: \href{https://www.suu.edu/suusa/voice}{https://www.suu.edu/suusa/voice} Likewise, you can learn more about SUUSA's Executive Council here (\href{https://www.suu.edu/suusa/executive-council/}{https://www.suu.edu/suusa/executive-council/}) and about individual SUUSA's Student Senators here (\href{https://www.suu.edu/suusa/senate/}{https://www.suu.edu/suusa/senate/})

\paragraph{Land Acknowledgement Statement:}
SUU wishes to acknowledge and honor the Indigenous communities of this region as original possessors, stewards, and inhabitants of this Too’veep (land), and recognize that the University is situated on the traditional homelands of the Nung’wu (Southern Paiute People). We recognize that these lands have deeply rooted spiritual, cultural, and historical significance to the Southern Paiutes. We offer gratitude for the land itself, for the collaborative and resilient nature of the Southern Paiute people, and for the continuous opportunity to study, learn, work, and build community on their homelands here today. Consistent with the University's ongoing commitment to equity, diversity, and inclusion, SUU works towards building meaningful relationships with Native Nations and Indigenous communities through academic pursuits, partnerships, historical recognitions, community service, and student success efforts.

\paragraph{Disclaimer:}
Information contained in this syllabus, other than the grading, late assignments, make up work and attendance policies, may be subject to change as deemed appropriate by the instructor.

\end{document}
